\chapter{总结与展望}
\label{cha:all:final}
\section{全文总结}
\label{sec:all:summary}
为提高贫纹理目标位姿追踪系统的精度和性能,本文采用基于物体三维模型与场景边缘图像配准的方法,有效降低追踪算法的计算复杂度,并且使用灰度相机即可完成高精度的位姿追踪。本文所提算法按照功能划分为
初始化系统以及追踪系统,针对各系统模块的功能开展了若干的相关研究和测试实验,测试实验结果显示本系统在复杂环境、不同模型复杂度下皆能完成高精度的位姿追踪,满足工业抓取等应用场合的基本要求。本文的主要研究成果
与总结如下:
\begin{enumerate}
    \item 实现追踪系统。针对贫纹理物体提出基于三维模型的位姿追踪算法,构建~DCM~目标函数,将追踪问题转换为寻优问题,实现物体模型边缘与场景图像边缘的配准,以解算位姿;
    改进寻优过程,使用~SLAM~算法中的图像点坐标与相机位姿的雅可比矩阵,替代原有数值导数,降低优化过程陷入局部最小值的概率,使得配准结果更加精准;
    针对遮挡问题提出自适应权重优化算法,使系统能够自动调整被遮挡部分的模型光栅点优化权重,减少被遮挡部分对位姿寻优引入的误差,降低图像干扰对追踪过程的影响,提高了系统稳定性。
    \item 制作检测与位姿真值数据库。使用追踪算法以及~CG~渲染技术对~4~种目标在不同场景下构建位姿数据库。完成检测数据库制作,使用追踪算法对真实场景视频中的目标物体进行位姿追踪,并保存物体~DCM~张量图,为物体的检测算法提供图像数据;
    完成位姿回归数据库构建,保存物体检测框的坐标信息,为后续平移向量回归算法提供真值数据;
    完成位姿真值数据库构建,使用~3ds Max~引擎搭建实验场景,利用该引擎计算物体位姿真值并渲染场景图像,用于对追踪算法进行精度测试,有效降低数据库制作成本,提高数据库真值的精度。
    \item 完成位姿初始化系统。本文实现基于机器学习方法的目标检测与平移向量回归方法。对检测数据库进行~NPD~特征提取,训练得到不同目标分类的检测随机森林;使用
    姿态回归数据库训练平移向量的回归森林,实现物体相对于相机坐标系的平移向量回归;最后针对所有目标物体构建姿态模板,使用位姿配准方法得到精确的初始位姿,完成自动初始化系统。
    \item 系统整合与优化。实现基于自适应权重计算的系统切换方法,使系统在追踪失败后能够自动重新初始化,解决物体在场景中消失后重新出现的问题,有效提高系统的稳定性;最后针对
    ~DCM~张量的两个计算过程:双向动态规划和平滑处理进行~GPU~加速,使用~CUDA~框架的并行计算方法,大幅降低~DCM~张量的计算耗时,提高系统的时效性。
    \item 对系统进行测试。使用真实场景的视频对初始化系统的各个模块进行测试,检测系统在~4~种目标物体上的正检率都在~$95\%$~左右,平移向量的估计误差在~3~毫米以内,使用模板匹配后初始化的精度达到
    追踪算法的精度,满足自动初始化的要求。在真实场景视频中对追踪算法完成测试,测试结果表明算法对背景的干扰具有强鲁棒性;在~Rigid Pose~数据集中的平均角度误差为~$1.835^\circ$,平移向量误差为
    ~$0.989$~毫米,均优于现有二维图像算法,精度与利用深度信息的~Dense~方法相当;在~CG~渲染视频中测试了追踪算法的精度,平均角度误差为~$0.846^\circ$,平移向量误差为~0.613;
    最后使用对比测试方法,验证了自适应权重优化对遮挡情况下的位姿寻优有显著提高。综合以上测试结果,验证了本文所提追踪算法能够在复杂环境下,对贫纹理物体实现高精度位姿追踪。
    \end{enumerate}

\section{未来工作展望}
限于时间、能力,本文的研究仍存在一些问题及尚需改进的部分,有待将来进行更深入的研究。下面结合本文的研究,就几个典型问题作出展望。
\begin{enumerate}
    \item 改进检测算法。现有目标检测算法多使用深度学习,并取得了不错的效果,可以考虑使用追踪算法得到大量的物体~DCM~张量图数据,使用该自制数据集作为模型训练样本以完成模型生成,
    进一步提高检测算法的正检率,减少多检以及误检的情况。
    \item 改进追踪算法。本文所提追踪算法在物体模型与场景边缘配准时,有可能出现多条模型边缘匹配至相同的场景边缘,导致出现匹配误差,该现象多出现在具有较大对称性的物体模型之上。
    后续可以研究带语义的场景边缘匹配方法,避免由于模型的对称性导致的场景边缘多匹配。
    \item 算法加速。本文已使用~CUDA~框架,对~DCM~张量计算中的双向动态规划以及平滑处理过程进行了加速,但单帧图像的计算时间仍难以满足实时要求,后续可以在图像边缘提取、方向离散化以及距离变换等步骤中研究加速计算的方法,
    进一步提高算法的实时性。
    \item 验证系统实用性。本文的工作主要在实验室环境下进行,虽然模拟了复杂背景以及光照的变换,但在实际使用情况中,可能存在相机抖动、物体运动速度过快的影响。因此需要针对实际应用场景,
    进一步验证追踪算法的性能,加入图像预处理等操作,提高系统的稳定性。
    \end{enumerate}


\label{sec:all:future_work}
