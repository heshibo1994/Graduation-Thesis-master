在此论文成稿之际,谨向我敬爱的导师岳继光教授和董延超副教授以最真诚的感谢与敬意!本文的选题、论文布局在岳老师的指导下完成,论文的创新方向以及细节实施均在董老师的负责下完成。
岳老师治学严谨,知识渊博,为人正直谦虚,与他的每一次谈话都让我受益匪浅。我从老师身上学到的不仅仅是知识,还有很多做人的道理,这些令我受用终身。董老师
有着丰富的项目经历和理论知识,能够快速定位研究中出现的问题以及痛点,合理安排我的研究方向以及时间,并训练我问题的凝练和解决方法,这对我处理实际问题的能力有了极大的提高,为日后进入工作打下了坚实的基础。
同时,岳老师和董老师不仅在学术上给我耐心的指导,还关心我的日常生活,让我在同济的三年研究生生涯过得顺利而又充实,再次向他们表示衷心的感谢。

感谢课题组的苏永清老师、吴继伟老师、李荣艳老师等,谢谢各位老师在论文开题时做出的指导。感谢师兄侯培鑫博士在学习和生活中的帮助,师兄极聪慧,知识面广,介绍给我非常多实用的工具以及技能,且师兄的
高标准促使我更加规范地完成每一件事。在侯博的主导下,我们一同完成并维护大论文的\LaTeX 模板,在此过程中我受益匪浅,再次感谢侯博,希望师兄在今后的工作生活中一切顺利。感谢师兄王森博、
徐晨剑以及施梁,三位师兄在我读研期间给予我非常多的指导和帮助,给我的研究生生活起到了榜样的作用,希望森博师兄顺利毕业,希望徐晨剑及施梁师兄工作顺利。
感谢同门徐刚、乔琪和张爽在日常学习中的齐心同行,也祝愿你们在日后的工作中取得好成绩。感谢张鲲鹏师兄、刘志刚师兄、汪胤师兄、唐丹旭师兄、王栗博士、吴琛浩博士,用丰富的生活和学习经验,使我更快
融入实验室集体。感谢师弟何士波,师妹林敏静在学术上的探讨和交流。感谢课题组成员,穆慧华、刘雪娇、刘金承、武新然、孙佳妮、李炅聪、冀玲玲、何洪志、寿佳鑫、王浩天、宁少淳带来的欢乐与帮助。此外,感谢室友闫智海,好友
胡钱珊、刘梦琪等,在学术上的交流讨论和生活中的娱乐和关心。

最后,感谢我的父母,是他们一直鼓励我、支持我,使我有了不断克服困难,奋勇前进的动力。

\vspace*{2\baselineskip}
\hspace*{11.6cm} 陈策 \vskip0em 
\hspace*{10.1cm} 2019 年 1 月于同济