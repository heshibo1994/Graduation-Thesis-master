\chapter{绪论}
\label{cha:chap1}
\section{研究背景和研究意义}
\label{sec:1.1}
基于
\subsection{研究背景}
\label{sec:1.1.1}
基于
\subsection{研究意义}
\label{sec:1.1.2}
基于
\section{国内外研究现状}
\label{sec:1.2}
基于
\subsection{SLAM的研究现状}
\label{sec:1.2.1}
即时定位与地图构建是让机器人在未知环境中持续地构建环境地图,并同时在地图中给自己定位。最早的SLAM技术还不是使用视觉的方法,
而是使用声波传感器或者激光以及惯性测量单元实现环境建模和自身定位。直到21世纪,Stephen Se等人首次使用图像的特征点实现视觉
SLAM\cite{se2002mobile},之后由Davison使用EKF框架实现了最早的单目实时SLAM系统\cite{davison2003real},奠定了单目系统
的基础;Davison在2007年成功实现基于单相机的纯视觉SLAM系统,算法的关键是在线建立2D点到3D点的映射关系,并且使用实时运动模
型估计相机的位置\cite{davison2007monoslam};Mur-Artal使用ORB特征点作为地图构建特征点,大幅度降低了点云的数量,并且使用
回环检测的方法使定位与建图的精度都大幅提升\cite{mur2015orb};随着硬件计算能力和数据储存的提升,提取目标深度信息的技术得
到了很大的发展,戚传江等人使用2D slam的解决方案,采用多传感器数据融合的方法,完成多自由度位姿检测[5],拓展了SLAM的应用场景;Whelan的实验通过使用体积融合的方法实现了实时大范围的稠密RGB-D的SLAM系统[6],
通过这个研究,使用SLAM系统作三维环境构建以及实时相机定位成为可能。

应用到目标跟踪领域,单纯点云集还是无法满足要求,因此需要将点云数据语义化,Reiger使用关系树的方法实现物体的语义识别[7],这项
技术对于目标跟踪是很重要的;之后Sarkar在Reiger的研究基础上结合FastSLAM的方法,使得识别速度更快,鲁棒性更强[8];Zhang, G等
人使用基于线条的SLAM算法[9]提高物体识别的准确率,该方法能够对物体的边沿与轮廓进行稳定的识别。

\subsection{三维重建的研究现状}
\label{sec:1.2.2}
基于
\subsection{基于视觉体积测算的研究现状}
\label{sec:1.2.3}
基于
\subsection{语义结构化地图的研究现状}
\label{sec:1.2.4}
基于
\section{待解决问题}
\label{sec:1.3}
基于
\section{主要研究内容和技术路线}
\label{sec:1.4}
基于
