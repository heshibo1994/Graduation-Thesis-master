\chapter{总结和展望}
\label{cha:chap6}
\section{全文总结}
\label{sec:6.1}
\label{sec:all:summary}
本文提出了基于纯视觉的无人机自主定位方法,能够使无人机在无GPS环境中围绕堆体进行自主飞行;本文基于传统三维重建方法,在重建实效性和精度方面提出了改进优化方法;本文提出了基于纯视觉的堆体体积测量方法。本文提出了面向无人机自主堆体体积测量的视觉定位及重建方法,可以构成一套完整的自动化测量系统。本文的主要研究成果与总结如下:
\begin{enumerate}
    \item 实现无人机自主定位。针对部分封闭,无GPS的环境中,构建一套无人机自主定位系统,通过结合二维码视觉标签的纯视觉方案,为无人机提供具有真实尺度的位姿数据,且坐标系统可以完全与真实世界对齐,使得无人机可以按照规定路线进行有效的自主飞行。结合二维码的视觉SLAM可以给无人机提供一个更加精度的定位数据,添加地图信息使整个系统更加稳定。
    \item 改进优化三维重建流程。使用结合视觉SLAM的方法对三维重建进行优化改良,三维重建本身可以较为简便地通过连续图像序列获取到场景的三维点云模型,但存在很严重的耗时问题以及点云精度低的问题。针对上述问题,提出了以SLAM对具备时序信息的连续帧进行处理,获取包含KeyFrame DataBase,匹配关系,相机位姿和环境地图等信息,作为先验知识提供给三维重建系统,提升原本系统的建模精度和速度。
    \item 实现视觉方法求解堆体提体积。本文提出基于纯视觉的估计方法代替传统测量仪器来对堆体进行体积测量,测量过程主要包括对堆体点云尺度的估计,堆体水平面方程的求解以及堆体体积的数学转化和计算。本方案在测量堆体体积时有着较好的准确性和稳定性,且流程简易,能够在多种场景下复用。
    \item 系统分模块测试。本文对无人机定位,三维重建改进方法,堆体体积测量三个方面进行分模块测试。其中无人机自主定位的精度可以控制在51.1$\%$以内,高度方向的误差在0.2m以内;改进后的三维重建点云模型在精度方面有了明显的提升,且可以很好解决实效性的问题;基于纯视觉的体积计算,整体测量误差精度可以控制2$\%$以内,且多次测量保证稳定性。
\end{enumerate}

\section{未来工作展望}
限于时间、能力,本文的研究仍存在一些问题以及尚需改进的部分,有待将来有更加深入的研究,以下结合本文的研究,就几个典型问题作出展望。
\begin{enumerate}
    \item 点云精度定性评价指标。本文对点云精度的评价指标多为定性处理,可以提出一种针对点云精度定量化分析的评价指标。
    \item 系统联合。本文提出的三个模块系统的联合较为初步,后续可以结合实际工程使得三个模块融合为完整系统,使得操作更加简易有效稳定。
    \item 堆体多场景测试。本文选用的堆体的对象为煤堆和沙堆,后续可以进一步对工农业生产中的多种类别堆体进行验证与测试。
    \end{enumerate}

    





   
