\tongjisetup{
  %******************************
  % 注意:
  %   1. 配置里面不要出现空行
  %   2. 不需要的配置信息可以删除
  %******************************
  %
  %=====
  % 秘级
  %=====
  secretlevel={保密},
  secretyear={2},
  %
  %=========
  % 中文信息
  %=========
  % 题目过长可以换行。
  ctitle={面向无人机自主堆体体积测量的视觉定位及重建方法研究},
  cheadingtitle={面向无人机自主堆体体积测量的视觉定位及重建方法研究},    %用于页眉的标题,不要换行
  cauthor={何士波},  
  studentnumber={1732940},
  cmajorfirst={工学},
  cmajorsecond={控制工程},
  cdepartment={电子与信息工程学院},
  csupervisor={岳继光~教授}, 
  % 如果没有副指导老师或者校外指导老师,把{}中内容留空即可,或者直接注释掉。
  cassosupervisor={董延超~副教授}, % 副指导老师
  % 日期自动使用当前时间,若需指定按如下方式修改:
  % cdate={超新星纪元},
  % 没有基金的话就注释掉吧。
  %cfunds={(国家杰出青年基金 (No.123456789) 支持)},
  %
  %=========
  % 英文信息
  %=========
  etitle={Visual localization and reconstruction method for autonomous bulk measurement of unmanned aerial vehicle}, 
  eauthor={He ShiBo},
  emajorfirst={Engineering},
  emajorsecond={Control Engineering},
  edepartment={College of Electronics and Information Engineering},
  %efunds={(Supported by the Natural Science Foundation of China for\\ Distinguished Young Scholars, Grant No.123456789)},    
  esupervisor={Prof. Yue Jiguang},
  eassosupervisor={V.P. Dong Yanchao}
  }

% 定义中英文摘要和关键字
\begin{cabstract}  
  在现代工业中,随着工业生产水平的提高,为了加强对堆体物料库存量的监控,针对堆体积测量的需求也逐步提升。传统测量方案一般都是选用激光雷达,测重地磅等复杂仪器来完成测量工作,这些仪器可以一定程度的完成测量工作并获取测量真实值。但依旧存在很多问题限制以上测量技术在行业内的广泛应用,例如测量仪器成本过高,安装复杂,单一场景内的仪器难以复用;测量过程需要人力参与,难以实现全自动化的测量流程;测量精度年久失衡,难以保证测量精度。针对以上情况,本文提出了一种在封闭电厂环境下,通过无人机的自主定位和飞行,对煤堆体进行图像采集工作,随后根据采集的图像进行三维重建得到高精度堆体点云模型,对点云模型估计体积值以获取堆体体积的完整测量流程。本文所提出方案完全基于计算机视觉实现,成本可控,且能在各种场景中复用。
  
  针对无人机的自主定位和飞行方法研究。本文提出在封闭无GPS的环境中,使用纯视觉的方式来完成无人机的自主定位和飞行工作。通过在场景中放置二维码,来解决视觉SLAM无法确定尺度的问题,并且结合SLAM坐标系和二维码坐标系对无人机提供真实世界坐标系下的位姿信息。
  
  针对连续图像的高精度堆体三维重建方法研究。本文提出一种结合SLAM结果的堆体三维重建方法,以有序图像作为SLAM输入,关键帧数据集作为输出提供给三维重建系统以获取高精度点云。解决传统三维重建图像匹配耗时,点云精度低的问题。
  
  针对堆体点云进行体积估计方法研究。本文提出一种基于计算机视觉方案的堆体体积估计方法,依次对生成的点云进行滤波获取高精度点云,估计三维点云的实际尺度,确定点云模型水平面等工作,最后根据纯点云信息估算场景中堆体的体积大小。
  
  基于真实环境下,对堆体体积测量进行分模块测试。实验结果表明,无人机的定位误差在高度方向可以控制在0.2m以内,整体定位精度在5$\%$以内,该精度完全可以提供给无人机循迹使用;改进后的三维重建模块则能加快匹配流程,获取较高精度的堆体三维点云,以满足体积测量的需求;最后的体积测量模块,可以估计出准确的尺度大小和水平面解析方程,通过对点云体积的求解获得堆体体积,测量误差可以控制在2$\%$以内。
\end{cabstract}

\ckeywords{立体视觉,SLAM,无人机定位, 三维重建,体积测量}

\begin{eabstract}
abstract
\end{eabstract}

\ekeywords{keywords}